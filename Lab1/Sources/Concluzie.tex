\section{Concluzie}
\tab In aceasta lucrare de laborator am facut cunostinta cu Version Control System care este o categorie de software tool-uri care ajuta o echipa de a organiza schimbarile codului in orice moment de timp. Version Control System pastreaza track-urile oricarei modificari, de aceea un avantaj al sau este eficienta, posibilitatea dezvoltatorilor de a lucra in paralel mai rapid. Asfel o echipa de dezvoltatori pot lucra individual la partea sa de lucru , dar la final sa le uneasca.De asemenea ofera posibilitatea de a reveni la o versiune anterioara, daca cea curenta nu ne convine,am creat un repositoriu local in care am lucrat,am creat fisiere,le-am editat,am creat branch-uri,am rezolvat conflicte utilizind un tool anume,am creat tag-uri si am facut commit-uri. Comenzile de baza si cele mai des folosite le-am aplicat in aceasta lucrare. 
\tab Am contientizat ca cu ajutorul GIT-ului putem lucra mult mai usor si eficient intr-o echipa.